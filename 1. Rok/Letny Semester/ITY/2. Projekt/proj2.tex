\documentclass[a4paper, 11pt, twocolumn]{article}

\usepackage[left=1.5cm, top=2.5cm, text={18cm, 25cm}]{geometry}
\usepackage{times}
\usepackage{amsthm, amssymb, amsmath}
\usepackage[czech]{babel}
\usepackage[utf8]{inputenc}
\usepackage[IL2]{fontenc}
\usepackage{enumitem}

\theoremstyle{definition}
\newtheorem{definition}{Definice}

\theoremstyle{plain}
\newtheorem{sentence}{Věta}
\newtheorem{algorithm}[definition]{Algoritmus}

\begin{document}
    \begin{titlepage}
        \begin{center}
            \Huge
            \textsc{Fakulta informačních technologií\\ Vysoké učení technické v~Brně}\\
            \vspace{\stretch{0.3}}
            \LARGE
                Typografie a publikování\,--\,2. projekt\\ Sazba dokumentů a matematických výrazů\\
            \vspace{\stretch{0.4}}
        \end{center}
        {\Large 2021 \hfill Martin Bublavý (xbubla02)}
    \end{titlepage}
	
    \section*{Úvod}
        V~této úloze si vyzkoušíme sazbu titulní strany, matematic\-kých vzorců, prostředí a dalších textových struktur obvyklých pro technicky zaměřené texty (například rovnice (\ref{eq_1}) nebo Definice \ref{Rozšířený zásobníkový automat} na straně \pageref{Rozšířený zásobníkový automat}). Rovněž si vyzkoušíme používání odkazů \verb+\ref+ a \verb+\pageref+.
        
        Na titulní straně je využito sázení nadpisu podle optického středu s~využitím zlatého řezu. Tento postup byl probírán na přednášce. Dále je použito odřádkování se zadanou relativní velikostí 0.4 em a 0.3 em.
        
        V~případě, že budete potřebovat vyjádřit matematickou konstrukci nebo symbol a nebude se Vám dařit jej nalézt v~samotném \LaTeX u, doporučuji prostudovat možnosti balíku maker \AmS -\LaTeX.
    
    \section{Matematický text}
        Nejprve se podíváme na sázení matematických symbolů a~výrazů v~plynulém textu včetně sazby definic a vět s~vy-užitím balíku \verb+amsthm+. Rovněž použijeme poznámku pod čarou s~použitím příkazu \verb+\footnote+. Někdy je vhodné použít konstrukci \verb+\mbox{}+, která říká, že text nemá být zalomen.

    \begin{definition}
        \label{Rozšířený zásobníkový automat}
        Rozšířený zásobníkový automat \emph{(RZA) je definován jako sedmice tvaru $A = (Q, \Sigma,\Gamma,\delta,q_{0},Z_{0},F)$, kde:}
        
        \begin{itemize}
            \item \emph{Q je konečná množina} vnitřních (řídicích) stavů,
            
            \item \emph{$\Sigma$ je konečná} vstupní abeceda,
            
            \item \emph{$\Gamma$ je konečná} zásobníková abeceda,
            
            \item \emph{$\delta$ je} přechodová funkce  $Q \times(\Sigma \cup\{\epsilon\}) \times \Gamma^{*} \rightarrow 2^{Q \times \Gamma^{*}}$,
            
            \item \emph{$q_{0} \in Q$ je} počáteční stav, $Z_{0} \in \Gamma$ \emph{je} startovací symbol zásobníku a $F \subseteq Q$ \emph{je množina} koncových stavů.
        \end{itemize} 
        
    \end{definition}
    
    Nechť $P = (Q,\Sigma,\Gamma,\delta,q_{0},Z_{0},F)$ je rozšířený zásobníkový automat. \emph{Konfigurací} nazveme trojici $(q,w,\alpha) \in Q \times \Sigma^{*} \times \Gamma^{*}$, kde $q$ je aktuální stav vnitřního řízení, $w$ je dosud nezpracovaná část vstupního řetězce a $\alpha = Z_{i_{1}} Z_{i_{2}} \ldots Z_{i_{k}}$ je obsah zásobníku\footnote{$Z_{i_{1}}$ je vrchol zásobníku}.
    
    \subsection{Podsekce obsahující větu a odkaz}
        \begin{definition}
            \label{Řetězec}
            Řetězec $w$ nad abecedou $\Sigma$ je přijat RZA \emph{A~jestliže ($q_{0}, w, Z_{0}$) $\underset{{A}}{\overset{*}{\vdash}}$ ($q_{F},\epsilon,\gamma$) pro nějaké $\gamma \in \Gamma^{*}$ a $q_{F} \in F$. Množinu $L(A)=\{w \mid w\ $  je přijat RZA $A\} \subseteq \Sigma^{*}$~nazýváme} jazyk přijímaný RZA $A$.
        \end{definition}
        
        Nyní si vyzkoušíme sazbu vět a důkazů opět s~použitím balíku \verb+amsthm+.
        \begin{sentence}
            Třída jazyků, které jsou přijímány ZA, odpovídá bezkontextovým jazykům.
        \end{sentence}
        \begin{proof}
            V~důkaze vyjdeme z~Definice \ref{Rozšířený zásobníkový automat} a \ref{Řetězec}. 
        \end{proof}
        
    \section{Rovnice a odkazy}
        Složitější matematické formulace sázíme mimo plynulý text. Lze umístit několik výrazů na jeden řádek, ale pak je třeba tyto vhodně oddělit, například příkazem \verb+\quad+.
        
        \[
        \sqrt[i]{x_{i}^{3}} \quad \textrm{kde }x_{i} \textrm{ je }i\textrm{-té sudé číslo splňující} \quad x_{i}^{x_{i}^{i^{2}}+2} \leq y_{i}^{x_{i}^{4}} 
        \]
        
        V~rovnici (\ref{eq_1}) jsou využity tři typy závorek s~různou explicitně definovanou velikostí.  
        
        \begin{eqnarray}
            \label{eq_1}
            & x &=\bigg[\Big\{ \big[a+b\big] * c\Big\}^d \oplus 2\bigg]^{3 / 2}\\
            & y &=\lim _{x \rightarrow \infty} \frac{\frac{1}{\log _{10} x}}{\sin ^{2} x+\cos ^{2} x} \nonumber
        \end{eqnarray}
        V~této větě vidíme, jak vypadá implicitní vysázení limity $\lim _{n \rightarrow \infty} f(n)$ v~normálním odstavci textu. Podobně je to i s~dalšími symboly jako $\prod_{i=1}^{n} 2^{i}$ či $\bigcap_{A \in \mathcal{B}} A$. V~případě vzorců $\underset{n \rightarrow \infty}{\lim} f(n)$ a $\underset{i=1}{\stackrel{n}{\prod}}2^{i}$ jsme si vynutili méně úspornou sazbu příkazem \verb+\limits+.
        
        \begin{eqnarray}
            \label{eq_2}
            \int_{b}^{a} g(x) \mathrm{d} x&=&-\int_{a}^{b} f(x) \mathrm{d} x
        \end{eqnarray}
    
    \section{Matice}
        Pro sázení matic se velmi často používá prostředí \verb+array+ a závorky (\verb+\left+, \verb+\right+).
        $$
        \left(\begin{array}{ccc}a-b & \widehat{\xi+\omega} & \pi \\ \vec{\mathbf{a}} & \overleftrightarrow{AC} & \hat{\beta}\end{array}\right)=1 \Longleftrightarrow \mathcal{Q}=\mathbb{R} \nonumber
        $$
        
        $$
        \mathbf{A}=\left\|
        \begin{array}{cccc} a_{11} & a_{12} & \ldots & a_{1 n} \\
        a_{21} & a_{22} & \ldots & a_{2 n} \\ 
        \vdots & \vdots & \ddots & \vdots \\
        a_{m 1} & a_{m 2} & \ldots & a_{m n} 
        \end{array}
        \right\|=\left|\begin{array}{cc}
        t & u~\\ 
        v~& w
        \end{array}\right| = tw - uv
        $$
        Prostředí \verb+array+ lze úspěšně využít i jinde.
        
        $$
    		\binom{n}{k}=
    		\left\{
    		\begin{array}{ll}
    			0 & \text{pro } k < 0 \text{ nebo } k > n\\
    			\frac{n!}{k! (n - k)!} & \text{pro } 0 \leq k \leq n. 
    		\end{array}
    		\right.
        $$
        
\end{document}